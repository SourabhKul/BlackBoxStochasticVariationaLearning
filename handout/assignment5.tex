\documentclass[11pt]{article}

\oddsidemargin=0in
\evensidemargin=0in
\textwidth=6.3in
\topmargin=-0.5in
\textheight=9in

\parindent=0in
%\pagestyle{empty}



%------------------------------------------------------------------
% PROBLEM, PART, AND POINT COUNTING...

% Create the problem number counter.  Initialize to zero.
\newcounter{problemnum}

% Specify that problems should be labeled with arabic numerals.
\renewcommand{\theproblemnum}{\arabic{problemnum}}


% Create the part-within-a-problem counter, "within" the problem counter.
% This counter resets to zero automatically every time the PROBLEMNUM counter
% is incremented.
\newcounter{partnum}[problemnum]

% Specify that parts should be labeled with lowercase letters.
\renewcommand{\thepartnum}{\arabic{problemnum}.\arabic{partnum}}

% Make a counter to keep track of total points assigned to problems...
\newcounter{totalpoints}

% Make counters to keep track of points for parts...
\newcounter{curprobpts}		% Points assigned for the problem as a whole.
\newcounter{totalparts}		% Total points assigned to the various parts.

% Make a counter to keep track of the number of points on each page...
\newcounter{pagepoints}
% This counter is reset each time a page is printed.

% This "program" keeps track of how many points appear on each page, so that
% the total can be printed on the page itself.  Points are added to the total
% for a page when the PART (not the problem) they are assigned to is specified.
% When a problem without parts appears, the PAGEPOINTS are incremented directly
% from the problem as a whole (CURPROBPTS).


%---------------------------------------------------------------------------


% The \problem environment first checks the information about the previous
% problem.  If no parts appeared (or if they were all assigned zero points,
% then it increments TOTALPOINTS directly from CURPROBPTS, the points assigned
% to the last problem as a whole.  If the last problem did contain parts, it
% checks to make sure that their point values total up to the correct sum.
% It then puts the problem number on the page, along with the points assigned
% to it.

\newenvironment{problem}[1]{
% STATEMENTS TO BE EXECUTED WHEN A NEW PROBLEM IS BEGUN:
%
% Increment the problem number counter, and set the current \ref value to that
% number.
\refstepcounter{problemnum}
%
% Add some vspace to separate from the last problem.
\vspace{0.15in} \par
%
\setcounter{curprobpts}{#1} \setcounter{totalparts}{0}	% Reset counters.
%
% Now put in the "announcement" on the page.
{\Large \bf \theproblemnum. \normalsize ({\it \arabic{curprobpts} point\null\ifnum \value{curprobpts} = 1\else s\fi}\/)}
}{
% STATEMENTS TO BE EXECUTED WHEN AN OLD PROBLEM IS ENDED:
%
% If no parts to problem, then increment TOTALPOINTS and PAGEPOINTS for the
% entire problem at once.
\ifnum \value{totalparts} = 0
	\addtocounter{totalpoints}{\value{curprobpts}}	% Add pts to total.
	\addtocounter{pagepoints}{\value{curprobpts}}	% Add pts to page total.
%
% If there were parts for the problem, then check to make sure they total up
% to the same number of points that the problem is worth. Issue a warning
% if not.
\else \ifnum \value{totalparts} = \value{curprobpts}
	\else \typeout{}
	\typeout{!!!!!!!   POINT ACCOUNTING ERROR   !!!!!!!!}
	\typeout{PROBLEM [\theproblemnum] WAS ALLOCATED \arabic{curprobpts} POINTS,}
	\typeout{BUT CONTAINS PARTS TOTALLING \arabic{totalparts} POINTS!}
	\typeout{}
	\fi
\fi
}


%---------------------------------------------------------------------------


% The \newpart command increments the part counter and displays an appropriate
% lowercase letter to mark the part.  It adds points to the point counter
% immediately.  If 0 points are specified, no point announcement is made.
% Otherwise, the announcement is in scriptsize italics.

\newcommand{\newpart}[1]
{
\refstepcounter{partnum}	% Set the current \ref value to the part number.
%\hspace{0.0in}		% Indent the part by a quarter inch.
%
% If points are to be printed for this problem (signaled by point value > 0),
% then put them in in scriptsize italics.
\ifnum #1 > 0
	\makebox[0.5in][l]{{\bf \thepartnum.} {\bf ({\it #1 pt\ifnum #1 = 1\else s\fi\/}) \,\,}}
\else
	\makebox[0.25in][l]{({\bf \thepartnum})}
\fi
%
\hspace{0.1in}		% Lead the material away from the part "number".
%
\addtocounter{totalparts}{#1}	% Add points to totalparts for this problem.
\addtocounter{pagepoints}{#1}	% Add points to total for this page.
\addtocounter{totalpoints}{#1}	% Add points to total for entire test.
}


%---------------------------------------------------------------------------



% Just in case you want to skip some numbers in your test...

\newcommand{\skipproblem}[1]{\addtocounter{problemnum}{#1}}



%---------------------------------------------------------------------------


% The \showpoints command simply gives a count of the total points read in up to
% the location at which the command is placed.  Typically, one places one
% \showpoints command at the end of the latex file, just prior to the
% \end{document} command.  It can appear elsewhere, however.

\newcommand{\showpoints}
{
\typeout{}
\typeout{====> A TOTAL OF \arabic{totalpoints} POINTS WERE READ.}
\typeout{}
}


%---------------------------------------------------------------------------



\usepackage{graphicx}
\usepackage[english]{babel}
\usepackage[latin1]{inputenc}
\usepackage{times}
\usepackage[T1]{fontenc}
\usepackage{amsmath}
\usepackage{amssymb}
\usepackage{subfigure}
\usepackage{algorithmic}
\usepackage{algorithm}
\usepackage{url}

\usepackage{xr}
\externaldocument{fa-notes}

\newcommand{\argmax}{\mathop{\arg\max}}
\newcommand{\deriv}[2]{\frac{\partial{#1}}{\partial {#2}} }
\newcommand{\dsep}{\mbox{dsep}}
\newcommand{\Pa}{\mathop{Pa}}
\newcommand{\ND}{\mbox{ND}}
\newcommand{\De}{\mbox{De}}
\newcommand{\Ch}{\mbox{Ch}}
\newcommand{\graphG}{{\mathcal{G}}}
\newcommand{\graphH}{{\mathcal{H}}}
\newcommand{\setA}{\mathcal{A}}
\newcommand{\setB}{\mathcal{B}}
\newcommand{\setS}{\mathcal{S}}
\newcommand{\setV}{\mathcal{V}}
\DeclareMathOperator*{\union}{\bigcup}
\DeclareMathOperator*{\intersection}{\bigcap}
\DeclareMathOperator*{\Val}{Val}
\newcommand{\mbf}[1]{{\mathbf{#1}}}
\newcommand{\eq}{\!=\!}
\newcommand{\cut}[1]{{}}

\begin{document}

%%%(change to appropriate class and semester)
{\centering
  \rule{6.3in}{2pt}
  \vspace{1em}
  \Large{
    CS688: Graphical Models - Spring 2018\\
    Assignment 5\\
  }
  \vspace{1em}
  Assigned: Tuesday, April 17th. Due: Tuesday, May 1st 11:59pm\\
  \vspace{0.1em}
  \rule{6.3in}{1.5pt}
}
\vspace{1pc}

\textbf{Getting Started:} You should complete the assignment using your own installation of Python 2.7. The only modules you are permitted to use in your implementations are Numpy, SciPy, and Autograd (\url{https://github.com/HIPS/autograd}). To get started with the code portions of the assignment, download the assignment archive from Moodle and unzip the file. The data files for this assignment are in the \textit{data} directory. Code templates are in the \texttt{code} directory.\\

\textbf{Deliverables:} This assignment has two types of deliverables: a report and code files.

\begin{itemize}
\item \textbf{Report: } The solution report will give your answers to the homework questions. Items that you should include in your report are marked with \textbf{(report)}. The maximum length of the report is 5 pages in 11 point font, including all figures and tables. You can use any software to create your report, but your report must be submitted in PDF format. You will upload the PDF of your report to Gradescope under \verb|HW05-Report| for grading. It is strongly recommended that you typeset your report. To assist with this if you wish to use Latex, the Latex source of the handout is also included in the homework archive.

\item \textbf{Code: } The second deliverable is your code. Items that you should include in your code are marked with \textbf{(code)}.  Your code must be Python 2.7 (no iPython notebooks, other formats, or code from other versions of Python). You will upload a zip file (not rar, bz2 or other compressed format) containing all of your code to Gradescope under \verb|HW05-Programming|.  When unzipped, your zip file should produce a directory called \verb|code|. Do not upload the data directory to Gradescope.

\end{itemize}
\vspace{0.5em}

\textbf{Academic Honesty Statement:} Copying solutions from external
sources (books, web pages, etc.) or other students is considered
cheating. Sharing your solutions with other students is also
considered cheating. Collaboration indistinguishable from copying is a violation 
of the course's collaboration policy and will be treated as cheating.
Any detected cheating will result in a grade of 0
on the assignment for all students involved, and potentially a grade
of F in the course.\\

\textbf{Introduction:} In this assignment, you will experiment with 
exact and approximate inference and learning for the factor analysis model. 
Factor analysis (FA) is a Gaussian latent variable model with Gaussian visible variables. 
It supports exact inference and marginal likelihood maximization with multiple
latent random variables. This allows for approximate inference methods
to be compared to exact inference methods. A tutorial introduction to
factor analysis is provided in the accompanying notes. The experiments
will focus on the use of factor analysis as a topic modeling tool. The implementations
will include both  exact and variational methods, including
stochastic black box methods. We will use the Autograd package
for automatic differentiation support.  It is recommended that you start
by reading the Autograd documentation, as well as the Autograd
back box stochastic variational inference examples: 
\url{https://github.com/HIPS/autograd}.
\\


\begin{problem}{25} \textbf{Derivations:} Perform the following derivations. All equation numbers refer to the accompanying FA notes. Show your work. (report)\\

\newpart{5} Starting from the definition of the model in terms of $P_{\theta}(\mbf{Z}=\mbf{z})$ and $P_{\theta}(\mbf{X}=\mbf{x}|\mbf{Z}=\mbf{z})$ shown in Equations \ref{eq:pz} and \ref{eq:pxgz}, derive the joint distribution for $P_{\theta}(\mbf{X}=\mbf{x},\mbf{Z}=\mbf{z})$ shown in Equation \ref{eq:pxz2}.\\

\newpart{10} Starting from the definition of the joint distribution for $P_{\theta}(\mbf{X}=\mbf{x},\mbf{Z}=\mbf{z})$ shown in Equation \ref{eq:pxz2}, derive the posterior distribution on $\mbf{Z}$, $P_{\theta}(\mbf{Z}=\mbf{z}|\mbf{X}=\mbf{x})$ shown in Equation \ref{eq:pzgx}.\\

\newpart{10} The true posterior on $\mbf{Z}$ is Gaussian with a general $K\times K$ covariance matrix. Suppose we
decide to approximate $P_{\theta}(\mbf{Z}=\mbf{z}|\mbf{X}=\mbf{x})$ using the distribution $Q_{\phi}(\mbf{Z}=\mbf{z})=\mathcal{N}(\mbf{z};\mbf{m},\mbf{S})$
where $\mbf{S}$ is a diagonal covariance matrix (i.e., $\mbf{S}_{kk}>0$, $\mbf{S}_{kj}=0$ for $k\neq j$). 
Derive the optimal variational parameter values $\mbf{m}$ and $\mbf{S}$ of $Q_{\phi}(\mbf{Z}=\mbf{z})$ by
analytically minimizing the objective function shown in Equation \ref{eq:KLPQ}.

\end{problem}


\begin{problem}{30} \textbf{Implementations:} Implement the following functions (code). You may add
	any additional methods you require. Note that several of the objective functions will need to be wrapped
	for use with an optimizer.\\

\newpart{5} Implement the function \verb|joint_likelihood|, which computes the joint log likelihood
of a collection of visible and latent variable values. See Equation \ref{eq:pxz}.\\

\newpart{5} Implement the function \verb|marginal_likelihood|, which computes the exact marginal log likelihood
of a collection of visible variable values. See Equation \ref{eq:px}.\\

\newpart{5} Implement the function \verb|svi_obj|, which computes stochastic
variational inference objective function shown in Equation \ref{eq:SVI}.\\

\newpart{5} Implement the function \verb|svl_obj|, which computes stochastic
variational learning objective function shown in Equation \ref{eq:SVL}.\\

\newpart{5} Implement the function \verb|infer|, which computes 
exact or approximate distribution over $Z$ given $\mbf{x}$. The
exact computation is given in Equation \ref{eq:pzgx}. 
For the approximate computation, use black box stochastic variational
inference by applying Autograd's automatic differentiation method to 
the \verb|svi_obj| function, and  then use the resulting  gradient in 
Autograd's \verb|adam| optimization method.\\

\newpart{5} Implement the function \verb|fit|, which learns the
model parameters using either direct marginal likelihood maximization,
or maximization of the stochastic variational learning objective.
In both cases, you can use Autograd's automatic differentiation method
to obtain the gradients needed for learning, and Autograd's
\verb|adam| optimization method. 
\end{problem}

\begin{problem}{25} \textbf{Inference Experiments:} In this question, you will experiment with the inference
methods implemented in the previous question. Provide any requested output in your report, and add the corresponding code to \verb|experiments.py|.\\

\newpart{5} Consider a simple instance of the FA model for $D=3$ and $K=2$. Let $\mbf{W}=[[1,1,0],[0,0,1]]$ and $\Psi$
be equal to the identity matrix. Run the \verb|infer| method on the data case $x=[[1,1,1]]$ in both exact and approximate modes. 
Report the exact posterior and the approximate posterior in terms of their means and covariance matrixes. 
Use the default number of samples in the variational approximation, and 1000 optimization iterations.\\

\newpart{5} Consider a second instance of the FA model for $D=3$ and $K=2$. Let $\mbf{W}=[[1,1,0],[0,1,1]]$ and $\Psi$
be equal to the identity matrix. Again run the \verb|infer| method on the data case $x=[[1,1,1]]$ in both exact and approximate modes. 
Report the exact posterior and the approximate posterior in terms of their means and covariance matrixes.
Use the default number of samples in the variational approximation, and 1000 optimization iterations.\\

\newpart{5} Explain why the variational approximation to the posterior $P(\mbf{Z}|\mbf{X}=\mbf{x})$ is very close to the 
the exact result using the model parameters from part 1, but significantly different from the exact result
using the model parameters from part 2.\\

\newpart{10} Explain how the model parameters $\mbf{W}$ and $\Psi$ affect the accuracy of the variational
approximation for this model in general. What types of settings of the parameters would result in the largest KL
divergence from the approximate posterior to the true posterior in the case where $D=3$ and $K=2$?
Support your answer with additional experiments.\\

\end{problem}

\begin{problem}{20} \textbf{Learning Experiments:}  In this question, you will experiment with 
learning the factor analysis model using the included data set \verb|nips_train.npy|
and \verb|nips_test.npy|. This data set consists of word counts from NIPS conference papers
up to the year 2016. Each file contains a word count array where row $n$
and column $d$ represent the number of times word $d$ was used in paper $n$. 
The file \verb|nips_vocab.npy| contains an array listing the words in the same 
order as the columns in the word count arrays (i.e.; word $d$ is given in position $d$).
Applying factor analysis to these data is a form of topic modeling. Each row of the learned factor loading
matrix $\mbf{W}$ is a set of weights over the words in the vocabulary. Large
weights within a row should indicate words that commonly co-occur in the same paper.\\

Provide any requested output in your report, and add the corresponding code to \verb|experiments.py|.\\

\newpart{5} Run your implementation of the \verb|fit| method in both exact and 
approximate mode on the training data with $K=10$. Report the average marginal log likelihood 
(as computed using the \verb|marginal_likelihood| method) of the
training data under the model fit using both exact and approximate learning. Next,
report the average marginal log likelihood of the
test data under the model fit using both exact and approximate learning.\\

\newpart{5} Using the model fit with exact learning,
determine the $10$ words with the highest weight under the learned 
$\mbf{W}$ matrix for each factor $k$. 
In a table, present the top 10 words along with their weights (sorted in decreasing order) for
each factor $k$. \\

\newpart{5} Using the model fit with approximate learning,
determine the $10$ words with the highest weight under the learned 
$\mbf{W}$ matrix for each factor $k$. 
In a table, present the top 10 words along with their weights (sorted in decreasing order) for
each factor $k$. \\
 
\newpart{5} Based on the results in parts 1 to 3, comment on how much you think the variational
approximation has decreased the effectiveness of learning.

\end{problem}

\showpoints
\end{document} 